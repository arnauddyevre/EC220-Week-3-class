%%%%%%%%%%%%%%%%%%%%%%%%%%%%%%%%%%%%%%%%%
% Beamer Presentation
% LaTeX Template
% Version 1.0 (10/11/12)
%
% This template has been downloaded from:
% http://www.LaTeXTemplates.com
%
% License:
% CC BY-NC-SA 3.0 (http://creativecommons.org/licenses/by-nc-sa/3.0/)
%
%%%%%%%%%%%%%%%%%%%%%%%%%%%%%%%%%%%%%%%%%

%----------------------------------------------------------------------------------------
%	PACKAGES AND THEMES
%----------------------------------------------------------------------------------------

\documentclass[aspectratio=169, 11pt]{beamer}

\mode<presentation> {

% The Beamer class comes with a number of default slide themes
% which change the colors and layouts of slides. Below this is a list
% of all the themes, uncomment each in turn to see what they look like.

%\usetheme{default}
%\usetheme{AnnArbor}
%\usetheme{Antibes}
%\usetheme{Bergen}
%\usetheme{Berkeley}
%\usetheme{Berlin}
%\usetheme{Boadilla}
%\usetheme{CambridgeUS}

\usetheme[progressbar=frametitle]{metropolis}
%\usetheme{Copenhagen}
%\usetheme{Darmstadt}
%\usetheme{Dresden}
%\usetheme{Frankfurt}
%\usetheme{Goettingen}
%\usetheme{Hannover}
%\usetheme{Ilmenau}
%\usetheme{JuanLesPins}
%\usetheme{Luebeck}
%\usetheme{Madrid}
%\usetheme{Malmoe}
%\usetheme{Marburg}
%\usetheme{Montpellier}
%\usetheme{PaloAlto}
%\usetheme{Pittsburgh}
%\usetheme{Rochester}
%\usetheme{Singapore}
%\usetheme{Szeged}
%\usetheme{Warsaw}

% As well as themes, the Beamer class has a number of color themes
% for any slide theme. Uncomment each of these in turn to see how it
% changes the colors of your current slide theme.

%\usecolortheme{albatross}
%\usecolortheme{beaver}
%\usecolortheme{beetle}
%\usecolortheme{crane}
%\usecolortheme{dolphin}
%\usecolortheme{dove}
%\usecolortheme{fly}
%\usecolortheme{lily}
%\usecolortheme{orchid}
%\usecolortheme{rose}
%\usecolortheme{seagull}
%\usecolortheme{seahorse}
%\usecolortheme{whale}
%\usecolortheme{wolverine}

%\setbeamertemplate{footline} % To remove the footer line in all slides uncomment this line
%\setbeamertemplate{footline}[page number] % To replace the footer line in all slides with a simple slide count uncomment this line

%\setbeamertemplate{navigation symbols}{} % To remove the navigation symbols from the bottom of all slides uncomment this line
}

\usepackage{graphicx} % Allows including images
\usepackage{booktabs} % Allows the use of \toprule, \midrule and \bottomrule in tables
\usepackage{amsmath}
\usepackage[english,brazil]{babel}
%\usepackage[latin1]{inputenc}
\usepackage{url,color}
\usepackage{subfigure}
\usepackage{amsthm,amsfonts,amssymb,amscd,amsxtra}
\usepackage{wrapfig}
\usepackage{soul}
\usepackage{xcolor}
\usepackage{array}
\usepackage{tikz}
\usetikzlibrary{trees}

% Set the overall layout of the tree
\tikzstyle{level 1}=[level distance=3.5cm, sibling distance=2cm]
\tikzstyle{level 2}=[level distance=3.5cm, sibling distance=1cm]

% Define styles for bags and leafs
\tikzstyle{bag} = [text width=4em, text centered]
\tikzstyle{end} = [circle, minimum width=3pt,fill, inner sep=0pt]

%Font
%\usefonttheme{professionalfonts} % using non standard fonts for beamer
%\usefonttheme{serif} % default family is serif
%\setmainfont{Liberation Serif}

%making the section titles appear before each section
\AtBeginSection[]
{
  \begin{frame}
    \frametitle{Table of Contents}
    \setbeamertemplate{section in toc}[sections numbered]
    \vspace{0.3cm}
    \tableofcontents[currentsection]
  \end{frame}
}



%----------------------------------------------------------------------------------------
%	TITLE PAGE
%----------------------------------------------------------------------------------------

\title[]{EC220 -- Introduction to Econometrics} % The short title appears at the bottom of every slide, the full title is only on the title page

\subtitle{Week 3}

\author{Arnaud Dy\`evre} % Your name
\institute[] % Your institution as it will appear on the bottom of every slide, may be shorthand to save space
{
}
\date{October 15\textsuperscript{th} 2020} % Date, can be changed to a custom date

\newtheorem{proposition}{Proposition}
\newtheorem{teo}{Theorem}
\newtheorem{exemplo}{Exemplo}
\newtheorem{corolario}{Corol\'{a}rio}

\newcommand{\R}{\mathbb{R}}
\newcommand{\x}{\textbf{x}}
\newcommand{\y}{\textbf{y}}
\renewcommand{\qedsymbol}{$\blacksquare$}
\newcommand{\dom}{\mathrm{dom}}
\newcommand{\ad}{\mathrm{ad}}
\newcommand{\gerado}{\mathrm{span}}


\begin{document}



\begin{frame}
\titlepage % Print the title page as the first slide
\end{frame}




\begin{frame}{Miscellaneous}

     
\begin{itemize}[<+- | alert@+>]
    \item Class is online for the rest of  term 
    \item Groups for problem sets
    \begin{itemize}
        \item New students added to the class
    \end{itemize}
    \item Does everyone have a \texttt{Stata} license now?
    \item Please submit your problem sets
    \begin{itemize}
        \item Important for your understanding of key concepts
        \item Useful for mastering \texttt{Stata}
    \end{itemize}
\end{itemize}


\end{frame}

\begin{frame}{Quick review of important notions}
    \begin{itemize}[<+->]
    \item Thinking about \textbf{counterfactuals}: Effect of insurance ($D_i$) on health ($Y_i$)
    
    % The sloped option gives rotated edge labels. Personally
    % I find sloped labels a bit difficult to read. Remove the sloped options
    % to get horizontal labels. 
\begin{tikzpicture}[grow=right, sloped]
\node[bag] {Sample pop.}
    child {
        node[bag] {Uninsured \\ $D_i = 0$}        
            child {
                node[end, label=right:
                    {$\textcolor{red}{Y_{0i} | D_i = 0}$}] {}
                edge from parent
                node[above] {\scriptsize{\textcolor{red}{Observed}}}
                node[below]  {}
            }
            child {
                node[end, label=right:
                    {$Y_{1i} | D_i = 0$}] {}
                edge from parent
                node[above] {\scriptsize{Unobserved}}
                node[below]  {}
            }
            edge from parent 
            node[above] {}
            node[below]  {}
    }
    child {
        node[bag] {Insured \\ $D_i = 1$}        
        child {
                node[end, label=right:
                    {$Y_{0i} | D_i = 1$}] {}
                edge from parent
                node[above] {\scriptsize{Unobserved}}
                node[below]  {}
            }
            child {
                node[end, label=right:
                    {$\textcolor{red}{Y_{1i} | D_i = 1}$}] {}
                edge from parent
                node[above] {\scriptsize{\textcolor{red}{Observed}}}
                node[below]  {}
            }
        edge from parent         
            node[above] {}
            node[below]  {}
    };
\end{tikzpicture}
    \item Observed difference $\mathbb{E}\left[\textcolor{red}{Y_{i} \mid D_{i}=1}\right]-\mathbb{E}\left[\textcolor{red}{Y_{i} \mid D_{i}=0}\right] = $ \\ 
    $ \underbrace{\mathbb{E}\left[\textcolor{red}{Y_{1 i} \mid D_{i}=1}\right]-\mathbb{E}\left[Y_{0 i} \mid D_{i}=1\right]}_{\text {average treatment effect on the treated }} + \underbrace{\mathbb{E}\left[Y_{0 i} \mid D_{i}=1\right]-\mathbb{E}\left[\textcolor{red}{Y_{0 i} \mid D_{i}=0}\right]}_{\text {selection bias }} $
\end{itemize}
\end{frame}

\begin{frame}{Quick review of important notions}
    \begin{itemize}[<+->]
        \item \textbf{Counterfactual outcomes} (=potential outcomes): What would have happened to someone under the scenarios where they had or had not been treated.
        \begin{itemize}
            \item Informally ``Any outcome that can happen''
        \end{itemize}
        \item \textbf{Observed outcomes}: Potential outcomes that are realised
        \begin{itemize}
            \item Informally, ``What actually happens''
        \end{itemize}
        \item \textbf{Selection bias}: Difference between the potential untreated outcomes ($Y_{0i}$) for treated ($D_i=1$) and untreated ($D_i=0$) individuals.
        \begin{itemize}
            \item Informally, ``Part of the difference in observed outcomes that is not explained by the treatment''
        \end{itemize}
    \end{itemize}
\end{frame}

\begin{frame}{Quick review of important notions}
    \begin{itemize}[<+->]
        \item \textbf{How does random assignment anihilate the selection bias?}: It makes $\mathbb{E}[Y_{0i}|D_i= 1] = \mathbb{E}[Y_{0i}|D_i= 0] $ 
        \begin{itemize}
            \item If $D_i$ is randomly assigned, then it is independent from the potential outcomes $\rightarrow$ $\mathbb{E}[Y_{i}|D_i] = \mathbb{E}[Y_{i}]$
            \item So $\underbrace{\mathbb{E}\left[Y_{0 i} \mid D_{i}=1\right]-\mathbb{E}\left[Y_{0 i} \mid D_{i}=0\right]}_{\text {selection bias }} = 0$ 
        \end{itemize}
        \item \textbf{Balance on observables}: $\bar{X}_{D_i=1} \approx \bar{X}_{D_i=0}$
        \begin{itemize}
            \item Informally, ``No large differences between the treated and the non-treated group''
            \item More on how to evaluate if differences are large in next problem set
        \end{itemize}
    \end{itemize}
\end{frame}

\begin{frame}{Problem set}
\begin{center}
    Please open \texttt{Stata} and follow along.
\end{center}
 
\end{frame}

\end{document}